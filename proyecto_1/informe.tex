%package list
\documentclass{article}
\usepackage[top=3cm, bottom=3cm, outer=3cm, inner=3cm]{geometry}
\usepackage{graphicx}
\usepackage{url}
%\usepackage{cite}
\usepackage{hyperref}
\usepackage{array}
%\usepackage{multicol}
\newcolumntype{x}[1]{>{\centering\arraybackslash\hspace{0pt}}p{#1}}
\usepackage{natbib}
\usepackage{pdfpages}
\usepackage{multirow}
\usepackage{multirow}
\usepackage[normalem]{ulem}
\useunder{\uline}{\ul}{}
\usepackage{amsmath}
\usepackage{float}
\usepackage{multicol}
\usepackage{subcaption}


% para tener url cortas en bib
\let\oldUrl\url
\renewcommand{\url}[1]{\href{#1}{Enlace}}
%

%%%%%%%%%%%%%%%%%%%%%%%%%%%%%%%%%%%%%%%%%%%%%%%%%%%%%%%%%%%%%%%%%%%%%%%%%%%%
%%%%%%%%%%%%%%%%%%%%%%%%%%%%%%%%%%%%%%%%%%%%%%%%%%%%%%%%%%%%%%%%%%%%%%%%%%%%
\newcommand{\csemail}{vmachacaa@ulasalle.edu.pe}
%\newcommand{\csdocente}{MSc. Vicente Enrique Machaca Arceda}
\newcommand{\csdocente}{
    Vicente Machaca Arceda \\
    Enzo Velásquez Lobatón \\
	Carlo Corrales Delgado\\
	Oscar Ramirez Valdez
	}
\newcommand{\cscurso}{Computación de Alto Desempeño}
\newcommand{\csuniversidad}{Universidad Nacional de San Agustín de Arequipa}
\newcommand{\csescuela}{Doctorado en Ciencias de la Computación}
\newcommand{\cspracnr}{03}
\newcommand{\cstema}{Centro de datos de Telefónica}
%%%%%%%%%%%%%%%%%%%%%%%%%%%%%%%%%%%%%%%%%%%%%%%%%%%%%%%%%%%%%%%%%%%%%%%%%%%%
%%%%%%%%%%%%%%%%%%%%%%%%%%%%%%%%%%%%%%%%%%%%%%%%%%%%%%%%%%%%%%%%%%%%%%%%%%%%


\usepackage[english,spanish]{babel}
\usepackage[utf8]{inputenc}
\AtBeginDocument{\selectlanguage{spanish}}
\renewcommand{\figurename}{Figura}
\renewcommand{\refname}{Referencias}
\renewcommand{\tablename}{Tabla} %esto no funciona cuando se usa babel
\AtBeginDocument{%
	\renewcommand\tablename{Tabla}
}

\usepackage{fancyhdr}
\pagestyle{fancy}
\fancyhf{}
\setlength{\headheight}{30pt}
\renewcommand{\headrulewidth}{1pt}
\renewcommand{\footrulewidth}{1pt}
\fancyhead[L]{\raisebox{-0.2\height}{\includegraphics[width=3cm]{img/logo_unsa}}}
\fancyhead[C]{}
\fancyhead[R]{\fontsize{7}{7}\selectfont	\csuniversidad \\ \csescuela \\ \textbf{\cscurso} }
\fancyfoot[L]{Vicente, Enzo, Oscar y Carlos}
\fancyfoot[C]{\cscurso}
\fancyfoot[R]{Página \thepage}


% para el codigo fuente
\usepackage{listings}
\usepackage{color}
\definecolor{dkgreen}{rgb}{0,0.6,0}
\definecolor{gray}{rgb}{0.5,0.5,0.5}
\definecolor{mauve}{rgb}{0.58,0,0.82}
\lstset{frame=tb,
	language=Python,
	aboveskip=3mm,
	belowskip=3mm,
	showstringspaces=false,
	columns=flexible,
	basicstyle={\small\ttfamily},
	numbers=none,
	numberstyle=\tiny\color{gray},
	keywordstyle=\color{blue},
	commentstyle=\color{dkgreen},
	stringstyle=\color{mauve},
	breaklines=true,
	breakatwhitespace=true,
	tabsize=3
}




\begin{document}
	
	
	
	
\begin{titlepage}
	
	\newcommand{\HRule}{\rule{\linewidth}{0.5mm}} % Defines a new command for the horizontal lines, change thickness here
	
	\center % Center everything on the page
	
	%----------------------------------------------------------------------------------------
	%	HEADING SECTIONS
	%----------------------------------------------------------------------------------------
	
	\textsc{\LARGE \csuniversidad}\\[1.5cm] % Name of your university/college
	\textsc{\Large \cscurso}\\[0.5cm] % Major heading such as course name
	%\textsc{\large Assignment 1}\\[0.5cm] % Minor heading such as course title
	
	%----------------------------------------------------------------------------------------
	%	TITLE SECTION
	%----------------------------------------------------------------------------------------
	
	\vspace{2cm}
	
	\HRule \\[0.4cm]
	{ \huge \bfseries \cstema}\\[0.4cm] % Title of your document
	\HRule \\[1.5cm]
	
	%----------------------------------------------------------------------------------------
	%	AUTHOR SECTION
	%----------------------------------------------------------------------------------------
	
	\begin{minipage}{0.4\textwidth}
		\begin{flushleft} \large
			\emph{Alumnos:}\\
			\csdocente
		\end{flushleft}
	\end{minipage}
	~
	\begin{minipage}{0.4\textwidth}
		\begin{flushright} \large
			\emph{Docente:} \\
			PhD. Alvaro Mamani Aliaga
		\end{flushright}
	\end{minipage}\\[2cm]
	
	% If you don't want a supervisor, uncomment the two lines below and remove the section above
	%\Large \emph{Author:}\\
	%John \textsc{Smith}\\[3cm] % Your name
	
	%----------------------------------------------------------------------------------------
	%	DATE SECTION
	%----------------------------------------------------------------------------------------
	
	{\large \today}\\[2cm] % Date, change the \today to a set date if you want to be precise
	
	%----------------------------------------------------------------------------------------
	%	LOGO SECTION
	%----------------------------------------------------------------------------------------
	
	\includegraphics[width=100px, keepaspectratio]{img/unsa}\\[1cm] % Include a department/university logo - this will require the graphicx package
	
	%----------------------------------------------------------------------------------------
	
	\vfill % Fill the rest of the page with whitespace
	
\end{titlepage}	
	
	
	

	
	
\tableofcontents
\newpage	
	

	
\section{Introducción}

\section{Marco teórico}
	
\subsection{Secuenciamiento de DNA}

\subsection{Secuenciemiento \textit{Next-Genration}}

\subsection{ Secuenciamiento \textit{Single-cell RNA}}

\subsection{NCBI y SRA toolkit}
	

% Yo(vicente) estoy haciedo la propuesta y resultados
\section{Propuesta}

En esta sección detallaremos el objetivo de la propuesta y los detalles técnicos para el desarrollo de la misma.

\subsection{Plataforma implementada}

En este trabajo se propone el desarrollo de una plataforma para el análisis de secuencias scRNA.  Por ejemplo en la Figura \ref{fig:analysis}, se presente las fases tradicionales de una análisis de secuencias de ADN (Next-generation). En este caso, vemos como se realiza el secuenciamiento de ADN/ARN, obteniendo miles y millones de lecturas cortas de la cadena de ADN. Entonces la idea es analizar estas secuencias con el fin de  saber que genes estan activos y como influye esto en el fenotipo de la especie. Por ejemplo, este análisis se suele realizar sobre celulas de tejido tumoral y se compara con otro experimiento que tomo muestras de celulas sanas, luego de una comparación se puede determinar cuales son los genes activos o inactivos según el tipo de tumor y enfermedad, esto nos ayuda a comprender mejor las enfermedades y mas aún detectarlas en fases tempranas. \\

Como se menciono antes, la propuesta se base en el desarrollo de una plataforma que permita un análisis de secuencias de ADN sobre un sistema distribuido. Realizar todo el análisis es costoso, debido a eso \textbf{en esta etapa se desarrollo en la verificación de calidad de las secuencias} de entrada. En proyectos futuros se completará la plataforma con mas funcionalidades. \\

\begin{figure}[H]
    \centering
    \includegraphics[width=0.6\textwidth]{img/proy/analysis2}
    \caption{Phases comunes realizadas en un análisis de secuencias de ADN (Next-generation).}
    \label{fig:analysis}
\end{figure}

\subsection{Herramientas utilizadas}

Para el desarrollo de la propuesta se ha utilizado las herramientas de la Tabla \ref{tab:tools}. Se escogio, Spark, debido a su versatilidad y la gestión de RDDs que permiten un desarrollo distribuido facil de implementar. Luego, se opto por utilizar Pyspark, porque la evaluación de los \textit{scripts} pueden hacerse desde el interprete de Python y incluso en Google Colab, esto es una ventaja porque reduce el tiempo de desarrollo.\\

\begin{table}[H]
	\centering
	\caption{Herramientas utilizadas para el proyecto}
	\label{tab:tools}
	\begin{tabular}{ll}
		\textbf{Herramienta} & \textbf{Version}
		\\ \hline
		Spark       & 3.2.0   \\
		Python      & 3.8.10  \\
		Pyspark     & 3.2.0  
	\end{tabular}
\end{table}

\subsection{Hardware}

En cuanto al hardware utilizado en el sistema distribuido, este es detallado en la Tabla \ref{tab:pcs}. Debido a la falta de recursos, solo se utilizo dos computadoras, una es un minicomputador Asus y una laptop Asus Taichi (ver Figura \ref{img:pcs})\\


\begin{table}[H]
	\centering
	\caption{Computadoras utilizadas en el sistema distribuído}
	\label{tab:pcs}
	\begin{tabular}{lp{8cm}}
		\textbf{Nombre de PC} & \textbf{Especificaciones}
		\\ \hline
		Desktop Asus      & Procesador i7 de séptima generación y 8GB de memoria RAM. Sistema operativo Linux.   \\
		Laptop Asus      & Procesador i5 de quinta generación y 4GB de memoria RAM. Sistema operativo Linux.\\
	\end{tabular}
\end{table}


 \begin{figure}[H]
	\centering
	\begin{multicols}{2}
		\includegraphics[width=\textwidth,height=0.2\textheight,keepaspectratio]{img/proy/pc_asus_1}\par 
		\includegraphics[width=\textwidth,height=0.2\textheight,keepaspectratio]{img/proy/pc_asus_2}\par 
	\end{multicols}
	\caption{Computadoras utilizadas en el sistema distribuído.}
	\label{img:pcs}
\end{figure}

\subsection{Funcionalidades}

Las funcionalidades de la propuesta son:
\begin{itemize}
	\item Conteo de la cantidad de secuencias. 
	\item Conteo total de las bases nitrogenadas. 
	\item Computo de la longitud de todas las secuencias. 
	\item Computo del promedio de las longitudes de las secuenias.
	\item Computo de la ocurrencia de cada base nitrogenada.
	\item Análisis de contenido por base.
\end{itemize}

\subsection{Implementación}

El código fuente de la propuesta está en \href{La implementación esta en un  repositorio de Github }{esté repositorio}. Lineas abajo, presentamos un extracto del archivo principal. \\

\begin{lstlisting}
# este codigo, realiza un pequenio analisis a vcarias secuencias scRNA, 
# se considero a ERR3014700. Se usa SRA toolkit para genera el archivo fasta.

from __future__ import print_function
from functools import wraps
import pyspark as spark
from pyspark import SparkConf
import time
from operator import add
import os 
from subprocess import STDOUT, check_call, check_output


class Fastq:
	def __init__(self, path:str) -> str:
		self.path = path
		self.stop_context()
		self.sc = spark.SparkContext.getOrCreate(conf=self.set_conf())
		self.data = self.sc.textFile(self.path)
	
	def stop_context(self):
		try:
			self.sc.stop()
			except:
		pass
	
	def set_conf(self):
		conf = SparkConf().setAppName("App")
		conf = (conf.setMaster('local[*]')
		.set('spark.executor.memory', '4G')
		.set('spark.driver.memory', '16G')
		.set('spark.driver.maxResultSize', '8G'))
		return conf

	def _logging(func):
	@wraps(func)
	def log_print(instance, *args, **kwargs):
		start = time.time()
		res = func(instance, *args, **kwargs)
		print("Finished Executing {}  in {}s!".format(func.__name__, time.time() - start))
		return res
		return log_print

	@_logging
	def get_data(self):
	return self.data

	@_logging
	def count_bases(self):
		seqs = self.extract_seq()
		seqs = seqs.flatMap(lambda line: list(line)) 
		seqs = seqs.map(lambda c: (c, 1))
		return seqs.reduceByKey(lambda a, b: a+b)#\
	
	
	@_logging
	def per_base_seq_content(self):
		seqs = self.extract_seq()
		seqs = seqs.flatMap(lambda line: list(line)) 
		seqs = seqs.map(lambda c: (c, 1))
		return seqs.reduceByKey(lambda a, b: a+b)#\
	
	@_logging
	def extract_seq(self):
		return self.data.filter(lambda x: x.isalpha())
	
	@_logging
	def get_lengths(self):
		seqs = self.extract_seq()
		return seqs.map(lambda x: len(x))
	
	def extract_qual(self):
		pass
	
	def extract_meta(self):
		pass


fasta = Fastq('/home/vicente/Documents/sratoolkit.2.11.3-ubuntu64/samples/ERR3014700.fastq')

# show first read
print("fasta head:", fasta.data.take(4))

# show read count
print("Read count:", fasta.data.count())

# extract sequences alone from the fastq file
seqs = fasta.extract_seq()

print("total sequences:", seqs.count())
print("sequences:", seqs.take(4))

# compute read lengths
lens = fasta.get_lengths()

# show the lengths of the first 10 reads
print("sequence lenghts:", lens.take(10))

# get the average read length
len_sum = lens.reduce(lambda x, y: x+y)
print("sequence mean lenght:", len_sum//lens.count())

# count base occurance
bases = fasta.count_bases()
print("Bases ocurrence:", bases.take(10))

import json
file = open("results.txt", "w")
results = {
'bases':  fasta.data.count(),
'total_seqs': seqs.count(),  
'seqs_len': lens.take(10),
'seqs_len_mean': len_sum//lens.count(),
'bases_ocurrence': bases.take(10)
}
file.write( json.dumps(results, indent=4) )

######################################################
# procesamos per base sequence content
#######################################################
seqs = fasta.extract_seq()

# get max, min, mean lenght
lens = fasta.get_lengths()
len_max = lens.reduce(lambda x, y: max(x, y))
len_min = lens.reduce(lambda x, y: min(x, y))
len_sum = lens.reduce(lambda x, y: x+y)
len_mean = len_sum//lens.count()

print(len_max, len_min, len_mean)

import numpy as np

# vectores de las 4 bases, para el conteo
#list_a = fasta.sc.parallelize(np.zeros( len_max ))
#list_c = fasta.sc.parallelize(np.zeros( len_max ))
#list_g = fasta.sc.parallelize(np.zeros( len_max ))
#ist_t = fasta.sc.parallelize(np.zeros( len_max ))

list_a = np.zeros( len_max )
list_c = np.zeros( len_max )
list_g = np.zeros( len_max )
list_t = np.zeros( len_max )

lens = fasta.get_lengths()
lens_np = np.array(lens.collect())

#for i in range( 3 ): # por cada base 
for i in range( len_max ):
if i < lens_np[i]:
acc_a = seqs.map(lambda x: x[i] if i < len(x)  else 'X' ).filter( lambda x: x=='A').count()
acc_c = seqs.map(lambda x: x[i] if i < len(x)  else 'X' ).filter( lambda x: x=='C').count()
acc_g = seqs.map(lambda x: x[i] if i < len(x)  else 'X' ).filter( lambda x: x=='G').count()
acc_t = seqs.map(lambda x: x[i] if i < len(x)  else 'X' ).filter( lambda x: x=='T').count()

list_a[i] = acc_a
list_c[i] = acc_c
list_g[i] = acc_g
list_t[i] = acc_t


n_seqs = seqs.count()
list_a = list_a/n_seqs
list_c = list_c/n_seqs
list_g = list_g/n_seqs
list_t = list_t/n_seqs

import matplotlib.pyplot as plt

plt.plot(range(len_max), list_a)
plt.plot(range(len_max), list_c)
plt.plot(range(len_max), list_g)
plt.plot(range(len_max), list_t)
#plt.show()
plt.savefig('per_base_content.png', bbox_inches='tight')
\end{lstlisting}




\section{Experimentos y Resultados}

Para evaluar el desempeño de la propuesta se evaluó las secuencias con el código \textbf{ERR3014700}, estas fueron descargadas de \href{https://www.ncbi.nlm.nih.gov/sra/ERR3014700}{NCBI}. Estás representan secuencias de \textit{Human betaherpesvirus}. El tamaño de estas secuencis es de 260.3 MB, en formato SRA. Luego fueron descomprimidos con la herramienta SRA de NCBI, como resultado se obtuvo las secuencias en formato FastaQ, en este caso el archivo llego a ocupar 590.8 MB. \\

Luego se levanto el sistema distribuído, se subio el \textit{script} a \textit{spark-submit} y se realizo el análisis. La plataforma retorna como resultado un archivo \textit{json} con detalle del análisis y un gráfico que detalla el análisis de contenido por base. Por ejemplo, para el experimiento mencionado anteriormente, se obtuvo el siguiente \textit{json}. \\

\begin{lstlisting}
{
	"bases": 1843156,
	"total_seqs": 462393,
	"seqs_len": [
		523,
		600,
		599,
		600,
		599,
		600,
		600,
		529,
		600,
		538
	],
	"seqs_len_mean": 564,
	"bases_ocurrence": [
		[
			"C",
			71397200
		],
		[
			"N",
			4054
		],
		[
			"D",
			3004
		],
		[
			"G",
			70502420
		],
		[
			"A",
			59888213
		],
		[
			"F",
			16160
		],
		[
			"T",
			59010438
		],
		[
			"B",
			203
		],
		[
			"E",
			4144
		]
	]
}
\end{lstlisting}


Para el caso del análisis de contenido por base, la plataforma generá el grafico de la Figura 	\ref{fig:analysis2}. Según este análisis, comprobamos que las primeras bases tiene un margen de error (las curvas caen y decaen), luego tambien verificamos que hay posiciones donde la Adenina tiene baja presencia (curva de color rojo),

\begin{figure}[H]
	\centering
	\includegraphics[width=0.6\textwidth]{img/proy/per_base_content}
	\caption{Anĺisis de contenido por base de las secuencias ERR3014700.}
	\label{fig:analysis2}
\end{figure}


\section{Conclusiones}

En este proyecto se ha desarrollado una herramienta distribuída que permite hacer el análisis masivo de grandes cantidades de lecturas de ADN (Next-generation sequencing). El objetivo fue demostrar que este análisis puede ser desarrollado utilizando Spark. \\

El proyecto se enfoco en el análisis de calidad de las lecturas de ADN, este es un paso crucial en cualquier experimento de Bioinformática. Como resultado, la propuesta obtiene estadísticas referentes a las ocurrencias de las bases nitrogenadas y además se desarrollo un análisis de contenido por base. \\
	
El proyecto esta es sus primeras etapas y en trabajos futuros se completará mas funcionalidades. El objetivo a largo plazo es desarrollar una herramienta distribuída que tenga las mismas actividades de las herramientas tradicionales.

\bibliographystyle{apalike}
%\bibliographystyle{IEEEtranN}
\bibliography{bibliography}

	
	
	
	
\end{document}